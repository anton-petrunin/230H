\documentclass{article}
\usepackage{amssymb, amsfonts, amsmath, amsthm}
\usepackage{hyperref}
\usepackage{enumerate}
\usepackage{epsfig,lpic,wrapfig}



\def\noi{\noindent}%  
\def\CC{\mathbb{C}}%  
\def\PP{\mathbb{P}}%  
\def\NN{\mathbb{N}}%  
\def\RR{\mathbb{R}}%  
\def\ZZ{\mathbb{Z}}% 
\def\QQ{\mathbb{Q}}% 

\def\cC{{\mathcal  C}}%  
\def\cF{{\mathcal  F}}%  
\def\cM{{\mathcal  M}}%  
\def\cS{{\mathcal  S}}%  
\def\cT{{\mathcal  T}}% 

\def\eps{\varepsilon}%  
\def\ge{\geqslant}%
\def\le{\leqslant}%
\def\phi{\varphi}%
\def\i{\subset}
\def\l{\left}
\def\r{\right}
\def\<{\langle}
\def\>{\rangle}
\def\:{\colon}

%operators
\def\Const{\operatorname{Const}}
\def\area{\operatorname{area}}
\def\vol{\operatorname{vol}}
\def\diam{\operatorname{diam}}
\def\codim{\operatorname{codim}}
\def\dim{\operatorname{dim}}
\def\dir{\operatorname{dir}}
\def\dist{\operatorname{dist}}
\def\pack{\operatorname{pack}}
\def\Ric{\operatorname{Ric}}
\def\op{\operatorname}
\def\parbf{\noindent\textbf}


\begin{document}

\title{Curves}
\author{}
\date{}
\maketitle

\parbf{Cycloid.} A cycloid is the curve traced by a point on the rim of a circular wheel as the wheel rolls along a straight line without slippage.

Assume the wheel is a unit circle, 
its rolls on the $x$-axis and in the upper half-plane with unit speed 
and the time $t=0$ the point on the rim is in the origin.
Let us find the position vector $\textbf{f}(t)$ of the point on the rim at time $t$.

Note that coordinates of the center of the wheel at time $t$ is $\textbf{c}(t)=(t,1)$.

Since there is no slippage the arc from the point $\textbf{d}(t)=(t,0)$ on the rim 
to the $\textbf{f}(t)$ has length $t$.
It follows that the angle the vectors $\textbf{w}(t)=\textbf{f}(t)-\textbf{c}(t)$ is clockwise rotation of the vector $(0,-1)=\textbf{d}(t)-\textbf{c}(t)$ by angle $t$.
Therefore 
\[\textbf{w}(t)=(-\sin t,-\cos t)\]
and
\[\textbf{f}(t)=\textbf{c}(t)+\textbf{w}(t)=(t-\sin t,1-\cos t).\]

\begin{center}
\begin{lpic}[t(4 mm),b(0 mm),r(0 mm),l(0 mm)]{pics/cycloid(1)}
\end{lpic}
\end{center}

\begin{wrapfigure}{o}{32 mm}
\begin{lpic}[t(-4 mm),b(0 mm),r(0 mm),l(0 mm)]{pics/pedal-curve(1)}
\lbl[rt]{22.5,15.5;$P$}
\end{lpic}
\end{wrapfigure}

\parbf{Pedal curve.} The pedal curve is traced by the orthogonal projection of a fixed point $P$ on the tangent lines of a given curve $\textbf{f}(t)$.


Write a parametric expression $\textbf{h}(t)$ for the pedal curve for the unit circle $\textbf{f}(t)=(\cos(t),\sin t)$ and the point $P=(1,0)$, so its position vector is $\textbf{i}$.

Denote by $\textbf{w}(t)$ the projection of $\textbf{v}(t)=\textbf{i}-\textbf{f}(t)$ to the tangent line at $\textbf{f}(t)$, so $\textbf{h}(t)=\textbf{f}(t)+\textbf{w}(t)$.

The velocity vector $\textbf{f}'(t)=(-\sin t,\cos t)$ is parallel to the tangent line at $\textbf{f}(t)$.
Note that $\|\textbf{f}'(t)\|=1$ for any $t$.
Therefore 
\begin{align*}
\textbf{w}(t)
&=
(\textbf{f}'(t)\cdot\textbf{v}(t))\textbf{f}'(t)
\\
&=(\textbf{f}'(t)\cdot(\textbf{i}-\textbf{f}(t)))\textbf{f}'(t)=
\\
&=(\sin^2 t,-\sin t\cos t).
\intertext{and} 
\textbf{h}(t)
&=\textbf{f}(t)+\textbf{w}(t)
\\
&=(\cos t+\sin^2t,\sin t-\sin t\cos t).
\end{align*}

\parbf{Involute.} Involute is a curve obtained from another given curve by attaching an imaginary taut string to the given curve and tracing its free end as it is wound onto that given curve. 

Write a parametric expression $\textbf{h}(t)$ for the involute for the unit circle $\textbf{f}(t)=(\cos(t),\sin t)$ starting at the point $P=(1,0)$, so its position vector is $\textbf{i}$.

Note that $\textbf{f}'(t)=(-\sin t,\cos t)$ and $\|\textbf{f}'(t)\|=1$.

Note that the vector $\textbf{w}(t)=\textbf{h}(t)-\textbf{f}(t)$ is tangent to the circle at $\textbf{f}(t)$ and since the length of the string equals to the length of unwound we get 
\[\textbf{h}(t)-\textbf{f}(t)=-t\textbf{f}'(t).\]
Therefore
\begin{align*}
\textbf{h}(t)&=\textbf{f}(t)-t\textbf{f}'(t)
\\
&=(\cos t+t\sin t,\sin t-t\cos t).
\end{align*}

\begin{center}
\begin{lpic}[t(4 mm),b(0 mm),r(0 mm),l(0 mm)]{pics/involute(1)}
\end{lpic}
\end{center}



\end{document}