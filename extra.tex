\documentclass{article}
\usepackage{amssymb, amsfonts, amsmath, amsthm}
\usepackage{hyperref}
\usepackage{enumerate}
\usepackage{graphicx}
\usepackage{cancel}
\usepackage{epsfig,lpic,wrapfig}
%\usepackage{ulem}
%\usepackage{url, multicol}


\def\noi{\noindent}%  
\def\CC{\mathbb{C}}%  
\def\PP{\mathbb{P}}%  
\def\NN{\mathbb{N}}%  
\def\RR{\mathbb{R}}%  
\def\ZZ{\mathbb{Z}}% 
\def\QQ{\mathbb{Q}}% 

\def\cC{{\mathcal  C}}%  
\def\cF{{\mathcal  F}}%  
\def\cM{{\mathcal  M}}%  
\def\cS{{\mathcal  S}}%  
\def\cT{{\mathcal  T}}% 

\def\eps{\varepsilon}%  
\def\ge{\geqslant}%
\def\le{\leqslant}%
\def\phi{\varphi}%
\def\i{\subset}
\def\l{\left}
\def\r{\right}
\def\<{\langle}
\def\>{\rangle}
\def\:{\colon}

%operators
\def\Const{\operatorname{Const}}
\def\area{\operatorname{area}}
\def\vol{\operatorname{vol}}
\def\diam{\operatorname{diam}}
\def\codim{\operatorname{codim}}
\def\dim{\operatorname{dim}}
\def\dir{\operatorname{dir}}
\def\dist{\operatorname{dist}}
\def\pack{\operatorname{pack}}
\def\Ric{\operatorname{Ric}}
\def\op{\operatorname}



\begin{document}

\title{Math 230H, Extra Credit Problems}
\author{}
\date{}
\maketitle

\begin{center}
{\small These are hard and interesting problems.
It might improve your score,\\
but should be used for fun. 
Only the first solution will be graded. 
\\ The solutions should 
be presents orally before November 18.}
\end{center}
\thispagestyle{empty}

\ 

\noi $1$. 
Given $n$ vectors $\mathbf{v}_1,\dots,\mathbf{v}_n$, 
denote by $A$ the sum $n$ numbers $\|\mathbf{v}_i\|^2$
and by $B$ the sum of $n(n-1)/2$ numbers 
$\|\mathbf{v}_i-\mathbf{v}_j\|^2$ for $1\le i<j\le n$.

Show that 
\[n A\ge B.\]

\ 

\noi
$2$. 
Two sets $\Gamma$ and $\Delta$ in the Euclidean space 
are called \emph{equidistant} if the distance function from $\Gamma$ is constant on $\Delta$ and the distance function from $\Delta$ is constant on $\Gamma$.
(For example, two concentric circles in one plane are equidistant.)

Describe all the pairs of equidistant circles in the Euclidean space.

\ 

\noi
$3a$. 
Given two vectors $\mathbf{x}_0$ and $\mathbf{v}$,
describe geometrically the set of solutions of the vector equation
\[\mathbf{x}\times(\mathbf{x}\times\mathbf{v})=\mathbf{x}_0\times(\mathbf{x}_0\times\mathbf{v})\]
with unknown vector  $\mathbf{x}$.

\ 

\noi
\cancel{$3b$.}\textit{(Solved)} 
Given two vectors $\mathbf{v}$ and $\mathbf{w}$,
describe geometrically the set of solutions of the vector equation
\[(\mathbf{x}\times\mathbf{v})\times(\mathbf{x}\times\mathbf{w})=\mathbf{0}\]
with unknown vector  $\mathbf{x}$.

\ 

\noi\cancel{$3c$.}\textit{(Solved)}
Given two vectors $\mathbf{v}$ and $\mathbf{w}$,
describe geometrically the set of solutions of the vector equation
\[\mathbf{x}\times\mathbf{v}=\|\mathbf{x}\|^2\mathbf{w}\]
with unknown vector  $\mathbf{x}$.

\ 

\noi
\cancel{$4$.}\textit{(Solved)} Given two points $P$ and $Q$ in the space consider the set of points $X$ such that the distance from $X$ to $P$ is twice larger than the distance from $X$ to $Q$.
Show that the set is formed by a sphere, find its radius and center in terms of $P$ and $Q$.

\ 

\noi
$5$. Show that equidistant set from two skew lines is a hyperbolic paraboloid.

\ 

\noi\cancel{$6$.}\textit{(Solved)}
 Show that there is no plane which is \emph{tangent} to the curve $\textbf{f}(t)=(t,t^2,t^3)$ at two distinct points. 

\emph{(A plane is called tangent to a curve $\textbf{f}(t)$ at point $\textbf{f}(t_0)$ it it contains the tangent line at $\textbf{f}(t_0)$.)}

\ 

\noi
$7$. Let $\textbf{f}(t)$ be a smooth curve in the plane.
Assume its curvature $\kappa(t)$ is increasing in $t$.
Show that the curve has \emph{no self-intersections};
that is, if $t_0\ne t_1$ then $\textbf{f}(t_0)\ne\textbf{f}(t_1)$.

\ 

\noi
$8$. 
Assume $f$ is a convex function of two variables. 
Let $\textbf{x}(t)=(x_1(t),x_2(t))$ and $\textbf{y}(t)=(y_1(t),y_2(t))$ be two plane curves such that 
\[\textbf{x}'(t)=\nabla f(\textbf{x}(t))
\quad\text{and}\quad
\textbf{y}'(t)=\nabla f(\textbf{y}(t))
\]
for any $t$.
Show that the function 
\[\ell(t)=\|\textbf{x}(t)-\textbf{y}(t)\|\]
is nondecreasing.

\emph{(A function $f(x,y)$ is called convex if its epigraph $z\ge f(x,y)$ is a convex set.)}

\ 


\begin{wrapfigure}{o}{21 mm}
\begin{lpic}[t(-4 mm),b(0 mm),r(0 mm),l(0 mm)]{pics/star-shaped-curve(1)}
\lbl[r]{12.5,11;$P$}
\end{lpic}
\end{wrapfigure}

\noi
$9$. Let $f$ be a smooth function of two variables. 
Assume that $\textbf{x}(t)$ is a smooth closed plane curve such that 
\[D^2_{\textbf{x}'(t)}f(\textbf{x}(t))=0\]
for any $t$ and
\[\tfrac{\partial^2 f}{\partial x^2}\tfrac{\partial^2 f}{\partial y^2}-(\tfrac{\partial^2 f}{\partial x\partial y})^2<0\]
at any point $\textbf{x}(t)$.

Show that the curve $\textbf{x}(t)$ can not be star-shaped.

\emph{(A closed plane curve is called star-shaped if every ray from some fixed point $P$ intersects the curve at a single point.)}
\end{document}






