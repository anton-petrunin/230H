\documentclass{article}
\usepackage{amssymb, amsfonts, amsmath, amsthm}
\usepackage{hyperref}
\usepackage{enumerate}
\usepackage{epsfig,lpic,wrapfig}



\def\noi{\noindent}%  
\def\CC{\mathbb{C}}%  
\def\PP{\mathbb{P}}%  
\def\NN{\mathbb{N}}%  
\def\RR{\mathbb{R}}%  
\def\ZZ{\mathbb{Z}}% 
\def\QQ{\mathbb{Q}}% 

\def\cC{{\mathcal  C}}%  
\def\cF{{\mathcal  F}}%  
\def\cM{{\mathcal  M}}%  
\def\cS{{\mathcal  S}}%  
\def\cT{{\mathcal  T}}% 

\def\eps{\varepsilon}%  
\def\ge{\geqslant}%
\def\le{\leqslant}%
\def\phi{\varphi}%
\def\i{\subset}
\def\l{\left}
\def\r{\right}
\def\<{\langle}
\def\>{\rangle}
\def\:{\colon}

%operators
\def\Const{\operatorname{Const}}
\def\area{\operatorname{area}}
\def\vol{\operatorname{vol}}
\def\diam{\operatorname{diam}}
\def\codim{\operatorname{codim}}
\def\dim{\operatorname{dim}}
\def\dir{\operatorname{dir}}
\def\dist{\operatorname{dist}}
\def\pack{\operatorname{pack}}
\def\Ric{\operatorname{Ric}}
\def\op{\operatorname}
\def\parbf{\noindent\mathbf}


\begin{document}

\title{Quizzes}
\author{}
\date{}
\maketitle

\section*{
Quiz 1
}

\noindent 1. Prove or give a counterexample: 
If $\mathbf{v}\cdot\mathbf{w} = 0$ for all $\mathbf{v}$, then
  $\mathbf{w} =\mathbf{0}$.
  
\noindent
\textit{Solution:}\\
If $\mathbf{v}\cdot\mathbf{w} = 0$ for all $\mathbf{v}$ then 
$$\|\mathbf{w}\|^2=\mathbf{w}\cdot\mathbf{w} = 0.$$\\ 
Therefore $\|\mathbf{w}\| = 0$ and $\mathbf{w}=\mathbf{0}$.

\ 

\noindent 2.
Assume $\|\mathbf{v}\|=\|\mathbf{w}\|$. 
Show that $(\mathbf{v}+\mathbf{w})\perp(\mathbf{v}-\mathbf{w})$.

\ 

\noindent
\textit{Solution:}\\
\begin{align*}
(\mathbf{v}+\mathbf{w})\cdot(\mathbf{v}-\mathbf{w})
&=\mathbf{v}\cdot\mathbf{v}-\mathbf{w}\cdot\mathbf{w}=
\\
&=\|\mathbf{v}\|^2-\|\mathbf{w}\|^2=
\\
&=0.
\end{align*}
Therefore $(\mathbf{v}+\mathbf{w})\perp(\mathbf{v}-\mathbf{w})$.

\section*{Quiz 2}

\noindent 1. If $\mathbf{v}$ and $\mathbf{w}$ are unit vectors in $\RR^3$, under what condition(s) would $\mathbf{v}\times\mathbf{w}$ also be a unit vector in $\RR^3$? 
Justify your answer.

\ 

\noindent
\textit{Solution:}
Denote by $\alpha$ the angle between $\mathbf{v}$ and $\mathbf{w}$. 
Then
\begin{align*}
\|\mathbf{v}\times\mathbf{w}\|&=\|\mathbf{v}\|\,\|\times\mathbf{w}\|\,\sin\alpha=
\\
&=\sin\alpha
\end{align*}Therefore 
\[\|\mathbf{v}\times\mathbf{w}\|\iff\sin\alpha=1\iff\alpha=\tfrac\pi2\iff\mathbf{v}\perp\mathbf{w}.\]

\ 

\noindent 2. 
Find the distance $d$ from the origin to the plane $x+2y+3z=4$.

\ 

\noindent
\textit{Solution:} Vector $\mathbf{n}=(1,2,3)$ is normal to the plane and the point with the position vector $\mathbf{x}_0=(4,0,0)$ belongs to the plane. 
The origin has position vector $\mathbf{0}=(0,0,0)$.
Therefore 
\begin{align*}
d&=\frac{\mathbf{n}\cdot(\mathbf{x}_0-\mathbf{0})}{\|\mathbf{n}\|}=
\\
&=\frac4{\sqrt{1+4+9}}=
\\
&=\frac4{\sqrt{14}}.
\end{align*}

\section*{
Quiz 3
}


\noindent 1. Determine if the given equation describes a sphere. 
If so, find its radius and center.
\[2x^2 + 2y^2 + 2z^2 + 4x + 4y + 4z + 44 = 0.\]


\ 

\noindent
\textit{Solution:} The equation can be rewritten as
\[(x+2)^2+(y+2)^2+(z+2)^2=-10.\]
It has no solutions since the left hand side is nonnegative and right hand side is negative; in particular it does not describe a sphere

\ 


\noindent 2. Assume $\rho(t)=t$, $\theta(t)=t$ and $\phi(t)=t$ be the spherical coordinates of the paricle, $0<t<\pi$. 
Find the position vector $\mathbf{r}(t)$ of the particle in the Cartesian coordinates 
and its velocity vector $\mathbf{v}(t)=\mathbf{r}'(t)$.


\ 

\noindent
\textit{Solution:} 

\[\mathbf{r}(t)=(t\sin t\cos t,t\sin^2 t, t\cos t).\]
Therefore
\[\mathbf{v}(t)=(\sin t\cos t+t(\cos^2t-\sin^2t),\sin^2 t+2t\sin t \cos t, \cos t-t\sin t).\]

\section*{
Quiz 4
}


\noindent 1. Calculate the arc length of $\mathbf{f}(t)=(3\cos 2t, 3\sin 2t, 3t)$ over the  interval $[0,\pi/2]$.



\ 


\noindent 2. Let $\mathbf{g}(s)$ be a smooth curve with arc length parametrization and $\kappa(s)$ be its curvature.
  Show that 
  \[\mathbf{g}\,'''(s)\cdot\mathbf{g}\,'(s)=-\kappa(s)^2\]
  for any $s$.
  
\ 

\noindent
\textit{Solution:} 

\begin{align*}
\|\mathbf{g}\,'(s)\|&=1,
\\
\mathbf{g}\,'(s)\cdot\mathbf{g}\,'(s) &=1,\quad \text{take the derivative of both sides}
\\
2\mathbf{g}\,''(s)\cdot\mathbf{g}\,'(s) &=0,\quad \text{once more}
\\
\mathbf{g}\,'''(s)\cdot\mathbf{g}\,'(s)+\mathbf{g}\,''(s)\cdot\mathbf{g}\,''(s) &=0,\quad \text{since}\quad \kappa(s)=\|\mathbf{g}\,''(s)\|,
\\
\mathbf{g}\,'''(s)\cdot\mathbf{g} &=-\kappa^2(s).
\end{align*}

\section*{
Quiz 5
}


\noindent 1. Parametrize the curve $\mathbf{f}(t) = (\cos 3t, \sin 3t, 4t)$, $t\in [0,\infty)$ by arc length.



\ 


\begin{wrapfigure}{o}{42 mm}
\begin{lpic}[t(-4 mm),b(0 mm),r(0 mm),l(0 mm)]{pics/pedal-curve(1)}
\lbl[rt]{22.5,15.5;$P$}
\end{lpic}
\end{wrapfigure}

\noindent 2. Recall that the pedal curve is traced by the orthogonal projection of a fixed point $P$ on the tangent lines of a given curve $\mathbf{f}(t)$.

Write a parametric expression for the pedal curve for the unit circle 
$\mathbf{f}(t)=(\cos t,\sin t)$ and the point $P=(1,0)$.

\ 

\noindent\textit{Solution:} 
The projection of $P$ to the tangent line is the sum of $\mathbf{f}(t)$ and the projection $\mathbf{w}(t)$ of $(1,0)-\mathbf{f}(t)$ in the direction of $\mathbf{f}'(t)$.
Since $\|\mathbf{f}'(t)\|=1$ we have
\[\mathbf{w}(t)=[((1,0)-\mathbf{f}(t))\cdot\mathbf{f}'(t)]\mathbf{f}'(t).\]
Therefore 
\begin{align*}
\mathbf{h}(t)&=\mathbf{f}(t)+\mathbf{w}(t)=
\\
&=\mathbf{f}(t)+[((1,0)-\mathbf{f}(t))\cdot\mathbf{f}'(t)]\mathbf{f}'(t)
\end{align*}
is a parametric equation for the pedal curve.

It remains to do the calculations:
\begin{align*}
\mathbf{f}(t)&=(\cos t,\sin t)
\\
\mathbf{f}'(t)&=(-\sin t,\cos t)
\\
((1,0)-\mathbf{f}(t))\cdot\mathbf{f}'(t)&=-\sin t
\\
\mathbf{h}(t)&=(\cos t+\sin^2 t,\sin t-\sin t\cos t).
\end{align*}

\section*{
Quiz 6
}


\noindent 1. Compute the gradient of \[f(x,y,z)=xyz.\]

\ 

\noindent 2. Find the equation of the tangent plane to the surface \[xyz=6\] at the point $P=(1,2,3)$.


\noindent\textit{Solution:} $f(x,y,z)=xyz$, $\nabla f(x,y,z)=(yz,xz,xy)$.
Therefore 
\[\nabla f(1,2,3)=(6,3,2)\]
and the equation is 
\[6(x-1)+3(y-2)+2(z-3)=0.\]

\section*{
Quiz 7
}


\noindent 1. Find three positive numbers $x$, $y$, $z$ whose sum is 10 such that $x^2 y^2 z$ is a maximum.



\ 




\noindent 2. Prove that if $(x_0,y_0)$ is a minimum point for a smooth function
  $f(x,y)$, then the tangent plane to the surface $z=f(x,y)$ at the point $(x_0,y_0,f(x_0,y_0))$ is parallel to the $xy$-plane.

\ 
  

\noindent\textit{Solution:}
Let $z_0=f(x_0,y_0)$ and $g(x,y,z)=f(x,y)-z$, so 
\[g(x,y,z)=0\] 
is the equation for our surface.

The tangent plane to the surface 
at $(x_0,y_0,z_0)$ is
\[(x-x_0,y-y_0,z-z_0)\cdot\nabla g(x_0,y_0,z_0)=0\]
or
\[(x-x_0)\tfrac{\partial f}{\partial x}(x_0,y_0)+(y-y_0)\tfrac{\partial f}{\partial x}(x_0,y_0)-(z-z_0)=0\]

Since $(x_0,y_0)$ is a minimum point
\[\tfrac{\partial f}{\partial x}(x_0,y_0)=\tfrac{\partial f}{\partial x}(x_0,y_0)=0.\]
Therefore the equation of tangent plane boils down to 
$z=z_0$
which is an equation of horizontal plane.

\section*{
Quiz 8
}

\noindent 1. Let $u$ and $v$ be twice-differentiable functions of a single variable, and let $c\ne 0$ be a constant.
  Show that $f(x,y)=u(x+cy)+v(x-cy)$ is a
  solution of the \emph{general one-dimensional wave equation}
\begin{displaymath}
   \frac{\partial^2 f}{\partial x^2} - \frac{1}{c^2}\,\frac{\partial^2 f}{\partial y^2} = 0 ~.
  \end{displaymath}





\noindent 2. Evaluate 
\[\iint\limits_{R} e^{x+y}\,dA,\]
where $R$ is the triangle with vertices $(0,0)$, $(2,0)$ and $(0,1)$.

\ 
  

\noindent\textit{Solution:}

\begin{align*}
\iint\limits_R e^{x+y}\,dA 
&=
\int\limits_0^1e^y\,dy\,\int\limits_0^{2(1-y)} e^x\,dx
\\
&=
\int\limits_0^1 e^y\,dy\, \left(e^x|_0^{2(1-y)}\right)
\\
&=
\int\limits_0^1 (e^{2-y}-e^y)\,dy
\\
&=
(-e^{2-y}-e^y)|_0^1
\\
&=e^2-2e+1.
\end{align*}

\section*{
Quiz 10
}

\noindent 1. 
Find the volume inside the elliptic cylinder $\frac{x^2}{4} + \frac{y^2}{9} \le 1$ for $0\le z\le 4$.


\ 



\noindent 2. 
Let $R$ be the annulus describe by the inequalities $1\le x^2+y^2\le 4$.
Evaluate 
\[\iint\limits_{R} e^{x^2+y^2}\,dA.\]
(\textit{Hint: Rewrite it in the polar coordinates $(\rho,\phi)$.})

\ 
  

\noindent\textit{Solution:} $x=\rho\cos\phi$, $y=\rho\sin\phi$.
Therefore
\begin{align*}
J(\rho,\phi)&=\det
\left(
\begin{matrix}
\cos\phi&-\rho\sin\phi
\\
\sin\phi&\rho\cos\phi
\end{matrix}
\right)
\\
&=\rho.
\end{align*}

The region $R$ parametrized by $1\le\rho\le 2$ and $0\le\phi\le2\pi$.
Therefore 

\begin{align*}
\iint\limits_{R} e^{x^2+y^2}\,dA
&=
\int\limits_0^{2\pi}\,d\phi\,\int\limits_1^2 e^{\rho^2}\rho\,d\rho
\\
&=
\tfrac12\int\limits_0^{2\pi}\,d\phi\,\int\limits_1^2 e^{\rho^2}d\rho^2
\\
&=\tfrac12 2\pi(e^4-1)=\pi(e^4-1)
\end{align*}

\section*{
Quiz 11
}

\noindent 1. Assume the plane region $R$ is defined by the inequalities $x^2+2y^2\le 2$. Rewrite double integral 
\[\iint\limits_R f(x,y)\,dA,\] 
as an iterated integral.


\ 




\noindent 2. 
Show that 
\[\int\limits_{-\infty}^{+\infty}\,\int\limits_{-\infty}^{+\infty}f(x,y)\,dx\,dy
=
\int\limits_{-\infty}^{+\infty}\,\int\limits_{-\infty}^{+\infty}f(x+y,x+2y)\,dx\,dy\]
For any smooth function $f(x,y)$ which vanish outside of a bounded region in the plane.

\ 

\noindent\textit{Solution:}
Consider the map $(x,y)\mapsto (x+y,x+2y)$.
Its Jacobian is
\[J(x,y)=\det\left(
\begin{matrix}
1&1
\\
1&2
\end{matrix}\right)=2-1=1.
\]
Note that the map $(x,y)\mapsto (x+y,x+2y)$ is a bijecton $\RR^2\to\RR^2$.

Therefore
\begin{align*}
\int\limits\limits_{-\infty}^{+\infty}\,\int\limits\limits_{-\infty}^{+\infty}f(x,y)\,dx\,dy
&=
\iint\limits_{\RR^2}f(x,y)\,dA
\\
&=\iint\limits_{\RR^2}f(x+y,x+2y)|J(x,y)|\,dA
\\
&=\int\limits\limits_{-\infty}^{+\infty}\,\int\limits\limits_{-\infty}^{+\infty}f(x+y,x+2y)\,dx\,dy
\end{align*}

\section*{
Quiz 12
}

\noindent 1. Is there a potential $F(x,y)$ for the field $\mathbf{f}(x,y) = y\,\mathbf{i} + x\,\mathbf{j}$? If so, find one.

\begin{wrapfigure}{o}{42 mm}
\begin{lpic}[t(-4 mm),b(0 mm),r(0 mm),l(0 mm)]{pics/deltoid(.75)}
%\lbl[rt]{22.5,15.5;$P$}
\end{lpic}
\end{wrapfigure}


\ 


\noindent 2. Express the area bounded by the deltoid curve 
\[(2\cos t+\cos 2t,2\sin t-\sin 2t),\quad t\in [0,2\pi]\] 
as a line integral.
(You do not have to evaluate the integral, but you may do so.
The deltoid curve is shown on the diagram; you can assume without proof that it has no self-intesections.)

\ 

\noindent\textit{Solution:}
\begin{align*}
\mathbf{r}(t)&=(2\cos t+\cos 2t,2\sin t-\sin 2t),
\\
\mathbf{r}'(t)&=(-2\sin t-2\sin 2t,2\cos t-2\cos 2t),
\\
\mathbf{f}(x,y)&=(0,x).
\end{align*}
Note that the curves $C$ goes counterclockwise around the region. Therefore
\begin{align*}
A
&=\oint\limits_C \mathbf{f}\cdot d\mathbf{r}
\\
&=\int\limits_0^{2\pi} (2\cos t+\cos 2t)(2\cos t-2\cos 2t)dt
\\
&=\int\limits_0^{2\pi} (4\cos^2 t-4\cos t\cos 2t-2\cos^2 2t)\, dt\end{align*}
We solved the problem; let us evaluate the integral
\begin{align*}
\quad &=\int\limits_0^{2\pi} (4\cos^2 t-2(\cos 3t+\cos t)-2\cos^2 2t)dt
\\
&=4\pi-2(0-0)-2\pi=2\pi.
\end{align*}



\section*{
Quiz 13
}


\noindent 1. Let $g(x)$ and $h(y)$ be differentiable functions, and let $\mathbf{f}(x,y)=h(y)\,\mathbf{i} + g(x)\,\mathbf{j}$.
Can $\mathbf{f}$ have a potential $F(x,y)$? 
If so, find it. 
You may assume that $F$ would be smooth.


\ 



\noindent 2. Evaluate 
\[\oint\limits_C (e^{\sin x}+y)\,dx+(e^{\cos y}-x)\,dy,\]
where $C$ is the unit circle $x^2+y^2=1$ traveled counterclockwise.
(\emph{Hint: use Green's formula}.)


\ 

\noindent\textit{Solution:}

\begin{align*}
P&= e^{\sin x}+y,
\\
Q&=e^{\cos y}-x,
\\
\frac{\partial Q}{\partial x}-\frac{\partial P}{\partial y}&=-2.
\end{align*}
Let us denote by $R$ the unit disc $x^2+y^2\le 1$.
By Green's formula,
\begin{align*}
\oint\limits_C (e^{\sin x}+y)\,dx+(e^{\cos y}-x)\,dy
&=
\iint\limits_R(-2)\,dA
\\
&=-2\, A(R)
\\
&=-2\pi.
\end{align*}









\section*{
Bicycle
}

Assume that the trajectory of the back wheel of an ideal bicycle is given by smooth plane curve $\mathbf{b}(t)$, here $t$ denotes time. Write an expression for the trajectory of the front wheel $\mathbf{f}(t)$. 
Show that the speed of the back wheel can not exceed the speed of the front wheel.
We assume that in the ideal bicycle the distance from back wheel and front wheel is fixed, let us denote it by $R$ and the back wheel always moves in the direction to the front wheel.

\ 
  

\noindent\textit{Solution:} Note that $\mathbf{u}(t)=\mathbf{b}'(t)/\|\mathbf{b}'(t)\|$ is the unit vector in the direction from $\mathbf{b}(t)$ to $\mathbf{f}(t)$.
Therefore
\begin{align*}
\mathbf{f}(t)&=\mathbf{b}(t)+R\mathbf{u}(t)=
\\
&=\mathbf{b}(t)+R\frac{\mathbf{b}'(t)}{\|\mathbf{b}'(t)\|}.
\end{align*}
and 
\begin{align*}
\mathbf{f}'(t)&=\mathbf{b}'(t)+R\mathbf{u}'(t).
\end{align*}

Since $\|\mathbf{u}(t)\|=1$ we have $\mathbf{u}'(t)\perp\mathbf{u}(t)\parallel \mathbf{b}'(t)$. By Pythagorean theorem,
\[\|\mathbf{f}'(t)\|^2=\|\mathbf{b}'(t)\|^2+R^2\|\mathbf{u}'(t)\|^2.\]
In particular, 
\[\|\mathbf{f}'(t)\|\ge \|\mathbf{b}'(t)\|\]
for any $t$.


\end{document}